\documentclass[a4paper,12pt]{article}
\begin{document}
\title{MAKERERE UNIVERSITY COLLEGE OF COMPUTING  AND INFORMATION TECHNOLOGY}
\author{By  SSENYONGA RICHARD SEMPEWO}

\maketitle
\section{Background of the college of computing and information technology}

The Faculty of Computing and Information Technology (CIT) was established by the University
Council on 19th January 2005 by upgrading the Institute of Computer Science into a faculty with
four departments of computer science, networks, information technology and information systems.
The Institute of Computer Science which was established by the University Council in 1985, grew out
of the University Computer Centre. CIT is mandated to run programmes in Computing, Information
Technology and related areas. Computing and Information Technology.

\section{Value and mission statements}

\subsection{Value statement}

he Faculty of Computing and Information Technology is an innovative and industryoriented
Faculty, pursuing excellence in teaching, learning, cutting edge value-added research and
consultancy, community outreach, as well as providing a vibrant student life.

\subsection{Mission statement}

To provide first class teaching, research and services in computing and ICT responsive
to national and international needs.

\section{Objectives}

The objectives of the college are to: -

a. Build human resource capacity in the area of Computing both the public and private sectors, especially in universities.

b. Develop research capacity in the area of Computing.

c. Address the increasing demand for graduates  in the area of computing.

d. Strengthen capacity and institutional building in the area of computing in tertiary institutions, private and public sectors.

e. Provide students with opportunities to develop skills in formulating, conducting and presenting their own scholarly research through the production of a research-based dissertations and publications.

f. Foster initiative and potential for independent self-study that will develop the students’ motivation and ability to continue updating their knowledge and skills after completion of the course of study in relation to scholarship and research

g. Enable the students to be able to demonstrate a critical awareness and reflection on research-based information as a basis for problem solving and practice in professional contexts.

h. Enable students to be able to demonstrate ability to interpret and report research findings in areas relevant to their
field of study.

i. Enable students to be able to demonstrate the ability to formulate research questions and problems, design and
carry out their own small scale research projects and present their findings orally and in writing.


\section{Computing}

Computing is concerned with the understanding, design and exploitation of computation and computer technology. It is a discipline that blends elegant theories (including those derived from a range of other disciplines such as mathematics, engineering, psychology, graphical design or well-founded experimental insight) with the solution of immediate practical problems; it combines the ethos of
the scholar with that of the professional; it underpins the development of both small scale and large systems that support organizational goals. A degree programme or a programme component in the case of joint degree, will count as lying within the area of computing if the existence of computers and associated technology is seen as a central driving force in its motivation. The following headings give a high-level characterisation of the whole area of computing, based on traditional hardware/ software and theory/ practice spectra, and including communication and interaction which spans across these areas:
\subsection{1. Hardware:}

• Computer architecture and construction
• Processor architecture
• Device-level issues and fabrication technology

\subsection{2. Software:}

• Programming languages
• Software tools and packages
• Computer applications
• Structuring of data and information
\subsection{3. Communication and interaction:}

• Computer networks, distributed systems
• Human-computer interaction, involving communication between computers and people
• Operating systems: the control of computers, resources and interaction

\subsection{4. Practice:}
• Problem identification and analysis
• Design, development, testing and evaluation
• Management and organization
• Professionalism and ethics
• Commercial and Industrial exploration

\subsection{5. Theory:}

• Algorithm design and analysis
• Formal methods and description techniques
• Modelling and frameworks
• Analysis, prediction and generalisation
• Human behaviour and performance
\section{Computing Careers}

Computing professionals might find themselves in a variety of environments in academia, research, industry, government, private and business organizations – analyzing problems for solutions, formulating and testing, using advanced communications or multi-media equipment, or working in teams for product development. Here’s a short list of research and vocational areas in computing (IEEE Computer Society [www.computer.org], www.computer.org/education/careers.htm).
1. Artificial Intelligence – Develop computers that simulate human learning and reasoning ability.

2. Computer Design and Engineering – Design new computer circuits, microchips, and other electronic components.

3. Computer Architecture – Design new computer instruction sets, and combine electronic or optical components to provide powerful but cost-effective computing.

4. Software Engineering – Develop methods for the production of software systems on time, within budget, and with few or no defects.

5. Computer Theory – Investigate the fundamental theories of how computers solve problems, and apply the results to other areas of computer science.

6. Operating Systems and Networks – Develop the basic software computers use to supervise themselves or to communicate with other computers.

7. Information Technology – Develop and manage information systems that support a business or organization.

8. Software Applications – Apply computing and technology to solving problems outside the computer field - in education or medicine

\section{Computing Equipment}

CIT  setup modern computing laboratories for all its undergraduate and graduate students and computer to student ratio now stands at 1:1. CIT has acquired more than 3000 computers and an assortment of ICT equipment under the Project on Building a Sustainable ICT Training Capacity in the four Public Universities in Uganda. In the new CIT building there are: 6 large computer labs each accommodating 700 students of which one is fully equipped; 4 smaller computer labs each accommodating 120 students of which two
are fully equipped. The new building has a Desktop publishing Unit and specialized labs such as Multimedia Lab, Advanced GIS Lab, Mobile Computing lab, Software Incubation lab, E-Learning Lab, Multi-Media Studio, software engineering lab; and Computer Engineering lab.

\section{Library}

CIT is equipped with a library that offers reading services and textbook loan services. The reading services that are offered within the library premises cater for a maximum of 50 occupants who can make use of the services on weekdays from 8am to 5pm. In order to offer maximum utilization of at least 5,000 volumes of textbooks maintained by the CIT librarian, loan (borrow and return) services are offered to both students and staff. The CIT library acquires textbooks from purchases made by the Faculty and Makerere University Main Library.  CIT will be equipped with a digital library so as to enhance access to academic content through the use of information and communication technology (ICTs). Makerere University Library services complement CIT library services by offering online library services that include an online catalogue and a variety of electronic resources that support research in computing and information technology.



\end {document}